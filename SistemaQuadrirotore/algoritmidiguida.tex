\section{Algoritmi di guida}

\begin{idee}
	\cite{ElikerKaram2018AOPf}
	\cite{baseTesi}
	\cite{Mendoza-SotoJoséLuis2018Cgpc}
	\cite{PathPlannigOverview}
	
	
	\cite{YangLiang2016SoR3} : 
	
	\cite {Literature3dPath} : 
	
	La necessità è quella di avere autonomia nello spostamento senza l'intervento dell'uomo, determinando autonomamente il percorso da seguire.
	Tra gli obbiettivi di pianificazione del percorso c'è anche quello di evitare gli ostacoli possibili per questioni di sicurezza.
	Nel contesto degli UAV il problema è un problema tridimensionale.
	
	La soluzione a questo tipo di problema è estremamente complessa.
	
	Si può suddividere il problema in 2 parti fondamentali:
	
	1. Percezione e modellazione dell'ambiente
	2. Applicazione dell'algoritmo di pianificazione
	
	Non è necessario però che la pianificazione si sempre accompagnata da una ottimizazione, alcune volte basta che questa colleghi i due punti nello spazio.
	Si distingue in pianificazione del persorso e pianificazione del percorso ottimo se si vuole ottimizzare una certa funzione di costo oppure no.
	
	Sussiste una differenza tra pianificazione del percorso e della traiettoria. La pianificazione del percorso cerca solo un a curva o una sequenza di curve che colleghino due punti nello spazio, mentre la pianificazione della traiettoria prevede di determinare anche come questa debba essere percorsa, descrivendo in modo cinematico e dinamico, attraverso la valutazione di questi nella ricerca della soluzione.
	
	Si possono classificare gli algoritmi di pianificazione dei percorsi in 5 categorie.
	
	1. Algoritmi basati sul campionamento:
		Questi algoritmi richiedono la conoscenza a priori di informazioni sull'ambiente e una rappresenzazione matematica di questo. Si prevede di dividere l'ambiente in nodi o celle o altre forme. Avviene poi una ricerca attraverso una ricerca casuale. Questa categoria si può suddividere a sua volta in altre 2 categorie : attive e passive. Attive  si intende algoritmi che esploraro rapidamente in modo casuale per trovare il percorso migliore. Passivi algoritmi che si occupano di cercare la soluzione da percorsi già presenti come su di una mappa.
		
	2. Algoritmi basati sui nodi
		Questi algoritmi sono basati sull'utilizzo di nodi appartenenti ad un grafo scomposto, ricercando le soluzioni gia eseguite.
		
	3. Algoritmi basati su modelli matematici
		Questi metodi modellano l'ambiente  e il sistema e il sistema e modellano la funzione di costo con i limiti e vincoli per ottenere la soluzione ottimale. Equazioni e disequazioni
		
	
	
\end{idee}