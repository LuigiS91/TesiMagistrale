\chapter{Revisione della letteratura}
\begin{commento}
A literature review is a comprehensive summary of previous research on a topic. The literature review surveys scholarly articles, books, and other sources relevant to a particular area of research. The review should enumerate, describe, summarize, objectively evaluate and clarify this previous research.  It should give a theoretical base for the research and help you (the author) determine the nature of your research.  The literature review acknowledges the work of previous researchers, and in so doing, assures the reader that your work has been well conceived.  It is assumed that by mentioning a previous work in the field of study, that the author has read, evaluated, and assimiliated that work into the work at hand.

A literature review creates a "landscape" for the reader, giving her or him a full understanding of the developments in the field.  This landscape informs the reader that the author has indeed assimilated all (or the vast majority of) previous, significant works in the field into her or his research. 

"In writing the literature review, the purpose is to convey to the reader what knowledge and ideas have been established on a topic, and what their strengths and weaknesses are. The literature review must be defined by a guiding concept (eg. your research objective, the problem or issue you are discussing, or your argumentative thesis). It is not just a descriptive list of the material available, or a set of summaries.(http://www.writing.utoronto.ca/advice/specific-types-of-writing/literature-review)

\end{commento}
\begin{idee}
	La principale fonte è la tesi di davide, essendo la continuazione naturale del suo lavoro.
	Cosa ha fatto nel suo lavoro.
	Nella libreria di gazebo è stato modificata la sezione dei rotori in modo da rispettare il lavoro della tesi di davide, essendo l'applicazione finale lo stesso drone.
	
	Il segnale preso dal controllore viene dal filtro di kalman di px4.
	Spiegazione di come funziona un filtro di kalman.
	
	Bisogna parlare dei controlli PID e SMC dalla sua bibliografia.
	
	Un'altra tesi è stata presa in considerazione per lo sviluppo del codice attraverso il tool. 
	
	Si è studiata la documentazione del tool.
	
	Si è utilizzato il codice sorgente è la relativa documentazione.
	
	Ci sono diverse tipologie di controllori applicati ai velivoli a pilotaggio remoto. 
	
	Quali tipologie di droni esistono.
	
	Quali sono le possibili applicazioni nel settore.
	
	Come è fatta la verifica dei prodotti nel settore industriale.
	
	Documentazione relativa a gazebo per l'istallazione e la modifica dei plug-in.
	
	Utilizzi di gazebo nel settore dei droni.
\end{idee}