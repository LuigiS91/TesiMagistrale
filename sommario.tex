\sommario
La tesi prevede lo sviluppo del software necessario per il governo di un velivolo a pilotaggio remoto, implementando diversi modelli di controllo e validando il controllore con simulazioni, in cui è stato considerato il software di bordo.

Data la complessità della generazione del software necessario alla guida e controllo dei velivoli di questo tipo, l'approccio più intuitivo, che permette la collaborazione tra diverse figure professionali quali programmatori, ingegneri aerospaziali e meccatronici, risulta essere sicuramente basata sull'utilizzo di una logica progettuale a modelli. Questo tipo di approccio permette di semplificare lo sviluppo del software, concentrando gli sforzi di chi si occupa della creazione dei modelli di controllo sugli aspetti effettivamente importanti della funzionalità di questi, lasciando la complessità della generazione di algoritmi e le relative basi di dati a chi ha maggiore competenza informatica, attraverso lo sviluppo degli strumenti che automatizzano la generazione del codice e la sua compilazione. 
Lo sviluppo del software effettuato in questo modo è accompagnato per ogni passaggio dalla corrispettiva fase di verifica: Model in the loop (MIL), Software in the loop (SIL), Prosessor in the loop (PIL), Hardware in the loop (HIL); aspetto derivato proprio dal classica metodologia di sviluppo del software utilizzante il modello denominato a V, e pienamente solidificato in molti settori industriali.

\'E stato scelto di utilizzare per l'implementazione dell'autopilota il dispositivo PixHawk con il relativo firmware PX4. Lo sviluppo del software avviene quindi specificatamente per questo sistema di prodotti e relativi firmware. Il codice è open-source, permettendo la possibilità di essere studiato e modificato. Il codice si trova ad uno stato abbastanza evoluto nello sviluppo, possedendo tutta una serie di funzionalità già sufficiente per gestire un velivolo a pilotaggio remoto. Nel caso però si voglia inserire un personale modello di guida e controllo o un nuovo stimatore, non sono previsti nativamente strumenti di modellazione, limitando lo sviluppo del software solo alle persone con una competenza informatica elevata, incrementando la complessità di progettazione stessa. In definitiva si ha difficoltà di creazione del codice sorgente partendo da un modello ,esistono comunque gli strumenti per verificare il codice tramite SIL e PIL.

In modo complementare all'utilizzo di PX4, si affianca l'utilizzo di un tool specifico per Simulink, chiamato "Embedded Coder Support Package
for PX4 Autopilots". Grazie a questo tool si permette la generazione del codice in modo automatico partendo dai modelli creati su Simulink e l'inserimento conforme del modulo prodotto all'interno del firmware e delle restanti funzionalità. Integrando quindi gli strumenti di compilazione presenti in PX4 si possono effettuare le simulazioni SIL e PIL, avvicinando lo sviluppo ad una approccio più industriale.

Per testare correttamente il funzionamento di un velivolo a pilotaggio remoto, occorre ricreare in un ambiente simulato il velivolo stesso e le interazioni principali con l'ambiente in cui opera. La scelta dell'ambiente simulato ricade nell'utilizzo di Gazebo. Come anticipato precedentemente la procedura di SIL è nativamente presente nel progetto di PX4. I simulatori selezionabili per questa fase di test sono svariati, il tool di Simulink però limita la scelta ad un simulatore esterno solamente, ovvero JMavSim. Attraverso una procedura particolare però si è reso possibile l'utilizzo di Gazebo, ambiente più versatile e completamente personalizzabile.

Sono stati creati due modelli di controllo delle diverse dinamiche del velivolo e successivamente alla corretta configurazione dei parametri dei controllori, sono stati messi a confronto attraverso simulazioni SIL. Per quando riguarda il controllo di posizione è stato utilizzato un semplice controllore Proportional Iintegral Derivative (PID), mentre per il controllo di quota e di assetto, sono stati utilizzati rispettivamente un controllore PID e un controllore Sliding Mode Control (SMC). L'algoritmo di guida invece prevede la generazione della traiettoria con profili di velocità trapezoidali.

Non è invece stato possibile effettuare correttamente la simulazione PIL a causa di alcuni problemi di comunicazione tra simulatore PixHawk. La trasmissione tra simulatore e autopilota avviene correttamente per quanto riguarda la gestione dei dati dei sensori, ciò non viene altrettanto per quanto riguarda i messaggi da fornire agli attuatori. Questo problema é probabilmente causato dalla differenza sostanziale che c'è tra l'applicazione generata dal tool di Simulink e la logica di base di PX4 nella gestione dei messaggi agli attuatori. Questo aspetto richiede ulteriori indagini in lavori futuri.

\begin{commento}
Inglese

The main purpose of this work is the development of the software necessary for the steering of a UAV (Unmanned Aerial Vehicle) with implementation of different control models and  the validation of the controller with software in the loop.

Given the complexity of the generation of the software necessary to guide and control UAVs (Unmanned Aerial Vehicle), the more intuitive approach, which allows collaboration between different professionals such as programmers, aerospace and mechatronics engineers, is certainly model based design. This type of approach makes it possible to simplify software development, concentrating the efforts the development of control models with focus on important aspects of their functionality, leaving the complexity of generating algorithms and related datastructure to those who have greater IT competence through the development of tools that automate the generation of the code and its compilation.
The software development carried out in this way is validated for each step by the corresponding verification phase: MIL (Model in the loop), SIL (Software in the loop), PIL (Prosessor in the loop), HIL (Hardware in the loop); derived from the classic software development V- model methodology, that is standard in many industrial sectors.

The implementation of the autopilot was made it on the PixHawk 4 device with the relative PX4 firmware. Therefore,the software development takes place specifically for these product systems and related firmware. The code is open-source, allowing the possibility to be studied and modified. The code is fully functional, possessing a whole series of features already sufficient to manage a remotely piloted aircraft. However, if you want to insert a personal GNC (Guidance , navigation and control) model or a new estimator, modeling tools not exist natively, limiting software development only to people with high IT (Information technology) skills, increasing the  overall design complexity itself. Ultimately, it is difficult to create the source code starting from a model, however there are tools included in che sources directory to verify the code through SIL (Software in the loop) and PIL (Prosessor in the loop).

MathWorks has made available a tool called "Embedded Coder Support Package
for PX4 Autopilots " that allows to compensate the aforementioned limitations. Thanks to this tool it is possible to generate the code automatically starting from the models created on Simulink and the insertion of the produced module into the firmware structure with the remaining useful functions. Therefore, integrating the compilation tools included in PX4, SIL (Software in the loop) and PIL (Prosessor in the loop) simulations can be performed, as a more industrial approach.

In order to properly test the operation of the UAV (Unmanned Aerial Vehicle), it is necessary to recreate in a simulated environment the aircraft itself and the main interactions with that. The choice of the simulated environment falls back on Gazebo. As previously mentioned, the SIL (Software in the loop) procedure is natively present in the PX4 project. The simulators that can be selected for this test phase are varied, the Simulink tool however limits the choice to an external simulator only that is JMavSim. However, through a particular procedure, it was possible to use Gazebo, a more versatile and customizable environment.

Two control models of the different dynamics of the aircraft were created and after the correct configuration of the controllers parameters, they were compared through SIL (Software in the loop) simulations. A simple PID (Proportional Iintegral Derivative) controller was used as position controller and a PID (Proportional Integral Derivative) controller and a SMC (Sliding Mode Control) controller respectively were used in order to attitude control. The GNC (Guidance, navigation and control) algorithm foresees a planning trajectory using trapezoidal velocity profile.

The PIL (Prosessor in the loop) simulation could not be performed correctly due to some communication problems between the PixHawk an the simulator. The transmission between the simulator and the autopilot takes place correctly as regards the management of sensor data, this was not the same as regards the messages to be provided to the actuators. This problem was probably caused by the substantial difference between the application generated by the Simulink tool and the basic logic of PX4 in managing messages to the actuators. This aspect requires further investigation in future work.

Abstract

The main purpose of this work is the development of the software necessary for the steering of a Unmanned Aerial Vehicle (UAV) with implementation of different control models and  the validation of the controller with software in the loop.
Given the complexity of the generation of the software necessary to guide and control Unmanned Aerial Vehicles (UAVs), the more intuitive approach, which allows collaboration between different professionals such as programmers, aerospace and mechatronics engineers, is certainly model based design.
The software development carried out in this way is validated for each step by the corresponding verification phase: Model in the loop (MIL), Software in the loop (SIL), Prosessor in the loop (SIL), Hardware in the loop (HIL); derived from the classic software development V-model methodology, that is standard in many industrial sectors.
The implementation of the autopilot was made it on the PixHawk 4 device with the relative PX4 open-source firmware. The code is fully functional, possessing a whole series of features already sufficient to manage a remotely piloted aircraft. However, if you want to insert a personal Guidance , navigation and control (GNC) model or a new estimator, modeling tools not exist natively, limiting software development only to people with high Information technology (IT) skills, increasing the  overall design complexity itself.
MathWorks has made available a tool called "Embedded Coder Support Package
for PX4 Autopilots " that allows to compensate the aforementioned limitations. Thanks to this tool it is possible to generate the code automatically starting from the models created on Simulink and the insertion of the produced module into the firmware structure with the remaining useful functions. Therefore, integrating the compilation tools included in PX4, SIL and PIL simulations can be performed, as a more industrial approach.
The choice of the simulated environment falls back on Gazebo.
Two control models of the different dynamics of the aircraft were created and after the correct configuration of the controllers parameters, they were compared through SIL simulations. A simple Proportional Iintegral Derivative (PID) controller was used as position controller and a PID controller and a Sliding Mode Control (SMC) controller respectively were used in order to attitude control. The GNC algorithm foresees a planning trajectory using trapezoidal velocity profile.
The PIL simulation could not be performed correctly due to some communication problems between the PixHawk an the simulator. The transmission between the simulator and the autopilot takes place correctly as regards the management of sensor data, this was not the same as regards the messages to be provided to the actuators. This problem was probably caused by the substantial difference between the application generated by the Simulink tool and the basic logic of PX4 in managing messages to the actuators. This aspect requires further investigation in future work.


ITALIANO


La tesi prevede lo sviluppo del software necessario per il governo di un velivolo a pilotaggio remoto, implementando diversi modelli di controllo e validando il controllore con simulazioni, in cui è stato considerato il software di bordo.
Data la complessità della generazione del software necessario alla guida e controllo dei velivoli di questo tipo, l'approccio più intuitivo, che permette la collaborazione tra diverse figure professionali quali programmatori, ingegneri aerospaziali e meccatronici, risulta essere sicuramente basata sull'utilizzo di una logica progettuale a modelli.
Lo sviluppo del software effettuato in questo modo è accompagnato per ogni passaggio dalla corrispettiva fase di verifica: Model in the loop (MIL), Software in the loop (SIL), Prosessor in the loop (PIL), Hardware in the loop (HIL); aspetto derivato proprio dal classica metodologia di sviluppo del software utilizzante il modello denominato a V, e pienamente solidificato in molti settori industriali.
E' stato scelto di utilizzare per l'implementazione dell'autopilota il dispositivo PixHawk con il relativo firmware open-source PX4. Il codice si trova ad uno stato evoluto nello sviluppo, possedendo tutta una serie di funzionalità già sufficiente per gestire un velivolo a pilotaggio remoto. Nel caso però si voglia inserire un personale modello di guida e controllo o un nuovo stimatore, non sono previsti nativamente strumenti di modellazione, limitando lo sviluppo del software solo alle persone con una competenza informatica elevata, incrementando la complessità di progettazione stessa.
In modo complementare all'utilizzo di PX4, si affianca l'utilizzo di un tool specifico per Simulink, chiamato "Embedded Coder Support Package
for PX4 Autopilots". Grazie a questo tool si permette la generazione del codice in modo automatico partendo dai modelli creati su Simulink e l'inserimento conforme del modulo prodotto all'interno del firmware e delle restanti funzionalità. Integrando quindi gli strumenti di compilazione presenti in PX4 si possono effettuare le simulazioni SIL e PIL, avvicinando lo sviluppo ad una approccio più industriale.
La scelta dell'ambiente simulato ricade nell'utilizzo di Gazebo.
Sono stati creati due modelli di controllo delle diverse dinamiche del velivolo e successivamente alla corretta configurazione dei parametri dei controllori, sono stati messi a confronto attraverso simulazioni SIL. Per quando riguarda il controllo di posizione è stato utilizzato un semplice controllore Proportional Integral Derivative (PID), mentre per il controllo di quota e di assetto, sono stati utilizzati rispettivamente un controllore PID e un controllore Sliding Mode Control (SMC). L'algoritmo di guida invece prevede la generazione della traiettoria con profili di velocità trapezoidali.
Non è invece stato possibile effettuare correttamente la simulazione PIL a causa di alcuni problemi di comunicazione tra simulatore PixHawk. La trasmissione tra simulatore e autopilota avviene correttamente per quanto riguarda la gestione dei dati dei sensori, ciò non viene altrettanto per quanto riguarda i messaggi da fornire agli attuatori. Questo problema é probabilmente causato dalla differenza sostanziale che c'è tra l'applicazione generata dal tool di Simulink e la logica di base di PX4 nella gestione dei messaggi agli attuatori. Questo aspetto richiede ulteriori indagini in lavori futuri.

\end{commento}