\begin{commento}
Capitolo sulla letteratura	
\end{commento}
\begin{idee}
	La principale fonte è la tesi di davide, essendo la continuazione naturale del suo lavoro.
	Cosa ha fatto nel suo lavoro.
	Nella libreria di gazebo è stato modificata la sezione dei rotori in modo da rispettare il lavoro della tesi di davide, essendo l'applicazione finale lo stesso drone.
	
	Il segnale preso dal controllore viene dal filtro di kalman di px4.
	Spiegazione di come funziona un filtro di kalman.
	
	Bisogna parlare dei controlli PID e SMC dalla sua bibliografia.
	
	Un'altra tesi è stata presa in considerazione per lo sviluppo del codice attraverso il tool. 
	
	Si è studiata la documentazione del tool.
	
	Si è utilizzato il codice sorgente è la relativa documentazione.
	
	Ci sono diverse tipologie di controllori applicati ai velivoli a pilotaggio remoto. 
	
	Quali tipologie di droni esistono.
	
	Quali sono le possibili applicazioni nel settore.
	
	Come è fatta la verifica dei prodotti nel settore industriale.
	
	Documentazione relativa a gazebo per l'istallazione e la modifica dei plug-in.
	
	Utilizzi di gazebo nel settore dei droni.
\end{idee}