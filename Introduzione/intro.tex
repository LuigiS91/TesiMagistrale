\chapter{Introduzione}
Lo scopo era di validare il codice generato.
Generare il codice attraverso code generation usando Simulink.
Usare il px4 e pixhawk con gazebo per simulare le manovre.
Valutare diverse manovre con differenti leggi di controllo.
Modellare le leggi di controllo e generare il codice per il pixhawk usando px4 attraverso l'utilizzo di Simulink.
Testare il software generato attraverso l'uso del simulatore. Caricare il programma generato nel pixhawk ed effettuare una simulazione simile per verificare.
Creare i modelli di controllo dell'autopilota.
Controllo pid e controllo smc.
Utilizzo delle funzionalità di generazione del codice messe a disposizioni da Simulink.

