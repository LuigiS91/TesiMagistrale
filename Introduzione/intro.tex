\chapter{Introduzione}
La tesi prevede lo sviluppo del software necessario per il governo di un velivolo a pilotaggio remoto, implementando diversi modelli di controllo e validando il controllore con diverse simulazioni.

Data la complessità della generazione del software necessario alla guida e controllo dei velivoli di questo tipo, l'approccio più intuitivo, che permette la collaborazione tra diverse figure professionali quali programmatori, ingegneri aerospaziali e meccatronici, risulta essere sicuramente basata sull'utilizzo di una logica progettuale a modelli. Questo tipo di approccio permette di semplificare lo sviluppo del software, concentrando gli sforzi di chi si occupa della creazione dei modelli di controllo sugli aspetti effettivamente importanti della funzionalità di questi, lasciando la complessità della generazione di algoritmi e le relative basi di dati a chi ha maggiore competenza informatica, attraverso lo sviluppo degli strumenti che automatizzano la generazione del codice e la sua compilazione. 
Lo sviluppo del software effettuato in questo modo è accompagnato per ogni passaggio dalla corrispettiva fase di verifica: MIL, SIL, PIL, HIL; aspetto derivato proprio dal classica metodologia di sviluppo del software utilizzante il modello denominato a V, e pienamente solidificato in molti settori industriali.

\'E stato scelto di utilizzare per l'implementazione dell'autopilota il dispositivo PixHawk con il relativo firmware PX4. Lo sviluppo del software avviene quindi specificatamente per questo ecosistema di prodotti e relativi firmware. Il codice del firmware è open-source, permettendo la possibilità di essere studiato e modificato. Il codice si trova ad uno stato abbastanza evoluto nello sviluppo, possedendo tutta una serie di funzionalità già sufficiente per gestire un velivolo a pilotaggio remoto. Nel caso però si voglia inserire un personale modello di guida e controllo o di un nuovo stimatore, non sono previsti nativamente strumenti di modellazione, limitando lo sviluppo del software solo alle persone con una competenza informatica alta, incrementando la complessità di progettazione stessa. In definitiva si ha difficoltà di creazione del codice sorgente partendo da un modello ,esistono comunque gli strumenti per verificare il codice tramite SIL e PIL.

In modo complementare a PX4, si affianca l'utilizzo di un tool specifico per Simulink, chiamato "Embedded Coder Support Package
for PX4 Autopilots". Grazie a questo tool si permette la generazione del codice in modo automatico partendo dai modelli creati su Simulink e l'inserimento conforme del modulo prodotto all'interno del firmware e delle restanti funzionalità. Integrando quindi gli strumenti di compilazione presenti in PX4 si possono effettuare le simulazioni SIL e PIL, avvicinando lo sviluppo ad una approccio più industriale.

Per testare correttamente il funzionamento di un velivolo a pilotaggio remoto, occorre ricreare in un ambiente simulato il velivolo stesso e le interazioni principali con l'ambiente in cui opera. La scelta dell'ambiente simulato ricade nell'utilizzo di Gazebo. Come anticipato precedentemente la procedura di SIL è nativamente presente nel progetto di PX4. I simulatori selezionabili per questa fase di test sono svariati, il tool di Simulink però limita la scelta ad un simulatore esterno solamente, ovvero jMavSim. Attraverso una procedura particolare però si è reso possibile l'utilizzo di Gazebo, ambiente più versatile e completamente personalizzabile.

Sono stati creati due modelli di controllo delle diverse dinamiche del velivolo e, successivamente alla corretta configurazione dei parametri, sono stati messi a confronto attraverso simulazioni SIL. Per quando riguarda il controllo di posizione è stato utilizzato un semplice controllore PID, mentre per il controllo di quota e di assetto, sono stati utilizzati rispettivamente un controllore PID e un controllore SMC. 

Non è invece stato possibile effettuare correttamente la simulazione PIL a causa di alcuni problemi di comunicazione tra simulatore e pixhawk. La trasmissione tra simulatore e autopilota avviene correttamente per quanto riguarda la gestione dei dati dei sensori, ciò non viene altrettanto per quanto riguarda i messaggi da fornire agli attuatori. Questo ultimo aspetto richiede ulteriori indagini in lavori futuri.