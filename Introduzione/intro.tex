\begin{commento}
Capitolo sulla letteratura	
\end{commento}
\begin{idee}
	La principale fonte è la tesi di davide, essendo la continuazione naturale del suo lavoro.
	Cosa ha fatto nel suo lavoro.
	Nella libreria di gazebo è stato modificata la sezione dei rotori in modo da rispettare il lavoro della tesi di davide, essendo l'applicazione finale lo stesso drone.
	
	Il segnale preso dal controllore viene dal filtro di kalman di px4.
	Spiegazione di come funziona un filtro di kalman.
	
	Bisogna parlare dei controlli PID e SMC dalla sua bibliografia.
	
	Un'altra tesi è stata presa in considerazione per lo sviluppo del codice attraverso il tool. 
	
	Si è studiata la documentazione del tool.
	
	Si è utilizzato il codice sorgente è la relativa documentazione.
	
	Ci sono diverse tipologie di controllori applicati ai velivoli a pilotaggio remoto. 
	
	Quali tipologie di droni esistono.
	
	Quali sono le possibili applicazioni nel settore.
	
	Come è fatta la verifica dei prodotti nel settore industriale.
	
	Documentazione relativa a gazebo per l'istallazione e la modifica dei plug-in.
	
	Utilizzi di gazebo nel settore dei droni.
\end{idee}
\begin{commento}
	Davide 
	I droni possono volare senza il comando diretto dell'uomo. 
	Funzionamento molto basilare.
	Diversi tipi di configurazione possibile (citazione agrawal2015multi).
	I droni sono comandati su tutti i gradi di libertà.
	Vantaggi e svantaggi dei droni. (ELENCARE QUALI)
	I droni possono essere impiegati in ambienti rischiosi e missinoi di svariata natura.
	Nascono quasi per gioco , solo secondariamente si è percepito il suo grande potenziale di impiego.
	Costano poco è sono una grande risolrsa su cui poter fare molta ricerca su modelli di controllo avanzati.
	Necessità di un controllo robusto anche in assenza di segnale GPS per applicazioni indoor.
	Necessità di avere una molteplicità di sensori per evitare gli ostacoli.
	
	bouabdallah2004pid
	
	bolandi2013attitude ... varia bibliografia
	
	Utilizzati due controlli per posizione e attitude. Perchè fare così?
	
	Controllori lineari e non lineari. Vantaggi e svantaggi. Utilizzo di quali? Descrivere i vari tipi di controllori che esistono.
	
	Applicazioni di queste leggi di controllo ad un caso reale con generazione del codice e verifica del funzionamento in ambiente simulato.
	Versatilità e possibilità di metodi alternativi come machine learning.
	falanga2018foldable possibilità di riconfigurazioni. panza2016tilt
	
	utilizzo di PX4 e sel suo stimatore.
	Parlare dei sensori.
	
	Suddivisione dei capitoli.
\end{commento}
\begin{commento}
	La nascita dei quadricotteri, si può far risalire agli arbori della nascita degli elicotteri.
	Il primo quadrirotore ad ala fissa fu costruito da George de Bothezat nel 1922. L'elicottero però non era facile da manovrare (Young, Warren R. The Helicopters. "The Epic of Flight". Chicago: Time-Life Books, 1982).
	Nel periodo successivo guazie allo sviluppo della'aeromodellistica degli anni 60' e allo sviluppo parallelo in ambito militare si arriva a sviluppare i primi droni commerciali di successo (PARROT AR Drone) governato interamente tramite wifi. Uso professionale e amatoriale. Presente il POV.molto semplice da usare. Main stream, necessarie normative restrittive. Riflessioni sulle infine possibilità di impiego. Sviluppo di sensori e algoritmi avanzati di controllo. inteligenze artificiali per evitare gli ostacoli.
\end{commento}

\begin{scaletta}
	Nascita dei quadrirotori e dei sistemi sapr. (Al contrario)
	Tipologie di droni e vantaggi delle architetture. escursus sui sensori e perchè sono utili
	Missioni in cui si possono usare
	tipo di controllo
	spiegazione dei capitoli
\end{scaletta}



