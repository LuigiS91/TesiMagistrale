\chapter{Introduzione}
La necessità di avere dei velivoli a pilotaggio remoto nasce inizialmente per scopi militari. Il primo velivolo radiocomandato fu infatti in De Haviland 82, adattato per l'addestramento della contraerea con un sistema di pilotaggio remoto attraverso radiocomando \cite{histoDrone}. \'E invece alla sperimentazione riguardante il mondo elicotteristico che si deve la comparsa delle prime configurazioni di quadricottero, un esempio il Jerome-de Bothezat Flying Octopus, velivolo costruito da George de Bothezat nel 1922 \cite{Young}. Ai tempi però questi tipi di configurazione avendo caratteristiche di controllabilità molto carenti e non riuscendo a soddisfare le caratteristiche di peso strutturale e potenze dei motori necessarie non permisero un successivo sviluppo a vantaggio delle configurazioni degli elicotteri così come la conosciamo adesso. Con il successivo avvento della modellistica negli anni '60, lo sviluppo di nuove tecnologie elettroniche e dei materiali, con relativo abbattimento dei costi, si sono potute superare le limitazioni del passato aprendo agli scenari applicativi più svariati oltre a quelli bellici, amatoriali e industriali. Nacquero quindi i primi modelli di droni commerciali.

Attualmente si parla in modo generico di multicottero, comprendendo in questa categoria molteplici configurazioni, tra le quali si è affermata maggiormente grazie alla versatilità e semplicità applicativa quella del quadricottero, configurazione utilizzata anche in questa tesi. Il funzionamento è basato sulla presenza di quattro rotori che generano il sostentamento producendo portanza dalla rotazione. Questi sono disposti in modo che sia anche possibile attraverso l'azionamento differenziale il controllo sul velivolo su tutti quanti i gradi di libertà, principalmente l'assetto e secondariamente a questa la posizione. I rotori sono installati su di una base rigida nella quale oltre al payload sono presenti un sistema di controllo necessario a stabilizzare e comandare il velivolo, i sensori necessari per la determinazione dello stato del velivolo e le interfacce in radiofrequenza per la trasmissione del radio comando e di messaggi più complessi per la navigazione e la telemetria. \'E proprio grazie alla presenza di questi avanzati moduli elettronici miniaturizzati ed estremamente integrati che è stato reso possibile l'utilizzo dei multicotteri altrimenti instabili e ingovernabili \cite{multi2015}. Il sistema infatti è caratterizzato dall'essere un sistema estremamente non lineare con dinamiche accoppiate. Inoltre, avendo in questo caso solo quattro attuazioni, esso risulta essere sottoattuato: meno controlli rispetto ai gradi di libertà \cite{nonlinear2008}.

Vengono riportati i principali vantaggi e svantaggi di questi sistemi \cite{DesTestCarm},\cite{irisquad}.
\paragraph{Vantaggi}
\begin{itemize}
	\item Capacità di mantenere una posizione nello spazio fissa e di decollare verticalmente.
	\item Grande manovrabilità. Il velivolo è in grado di cambiare repentinamente il proprio assetto e la propria velocità.
	\item Sistema molto semplice da costruire e mantenere.
\end{itemize}
\paragraph{Svantaggi}
\begin{itemize}
	\item La durata della batteria è limitata. Risulta limitata l'autonomia di volo.
	\item Il sistema è sottoattuato e non lineare. Questo si traduce in una difficoltà nel essere controllato.
	\item Il payload imbarcabile è relativamente basso.
\end{itemize}

Un aspetto importate da considerare è la presenza dei sensori, senza di questi non sarebbe possibile ricavare attraverso filtraggio e stima, lo stato del velivolo e quindi determinare il controllo da attuare per ottenere la risposta, in termini di posizione e velocità o traiettoria desiderata. I tipici sensori installati sono quelli che permettono la determinazione delle accelerazioni sui tre assi e angolari. Teoricamente queste sei misure se integrate sarebbero sufficienti, ma a causa della intrinseca presenza di rumore ed errori non del tutto eliminabili, si adoperano ulteriori sensori. L'antenna di ricezione del Sistema di posizionamento globale (GPS), un altrimetro barometrico e magnetometri, possono compensare i difetti di misurazione, attraverso filtraggio e composizione delle misure \cite{KoksalN2018ALQA}. Nell'applicazione specifica trattata in questa tesi il segnale GPS non è da considerarsi affidabile in quanto, viene ripreso in parte il lavoro di ricerca svolto al Politecnico di Torino \cite{DesTestCarm}, \cite{baseTesi}.

\todo[inline]{lavori simili}
\begin{commento}
citare Development of Hardware-in-the-Loop Simulation Based on Gazebo
	and Pixhawk for Unmanned Aerial Vehicles : assenza di simulink
citare il lavoro di Davide
citare la tesi naval con luso di simulink
citare Implementation of Sliding Mode Fault Tolerant Control on the IRIS+
Quadrotor
\end{commento}

\todo[inline]{motivazione del lavoro}
\todo[inline]{obbiettivo del lavoro}
\todo[inline]{Perchè è importate fare SITL e PITL invece che flight test}
Come riportato nella stessa guida di PX4, le simulazioni sono il modo più sicuro e rapido per testare il codice, prima di avere il vero e proprio quadricottero. Dal settore automotive si può capire l'importanza di un approccio model based con corrispettive fasi di validazione, citando Model-based testing of Automotive System, vantaggi nella progettazione semplificando la descrizione del problema e nell'automazione della fase di testing del prodotto. 



\section{Organigramma della tesi}
La tesi è suddivisa, oltre a questo capitolo introduttivo, in tre capitoli:
\begin{itemize}
	\item \textbf{Sistema Quadrirotore :} In questo capitolo verrà trattato nel dettaglio il sistema quadrirotore. Nello specifico verranno considerati il modello matematico, le leggi di controllo adottate e gli algoritmi di guida.
	\item \textbf{Descrizione Autopilota :} Verranno trattati i dettagli del sistema scelto per l'implementazione. Varranno descritti in questo capitolo i dettagli del drone, l'autopilota scelto con il corrispettivo firmware e gli strumenti messi a disposizione per i test, con le modifiche effettuate.
	\item \textbf{Simulazioni : } In questo ultimo capitolo verranno messi a confronto attraverso le simulazioni due differenti sistemi di controllo e le conclusioni. L'ultima parte del capitolo riguarderà il lavoro svolto sulle simulazioni PIL e i problemi riscontrati e possibili studi futuri.
\end{itemize}